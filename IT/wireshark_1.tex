\documentclass[a4paper]{article}
\usepackage{geometry}
%\usepackage[nostamp,tikz,svg]{moodle}
\usepackage[handout,nostamp,tikz,svg]{moodle}
\pagestyle{empty}
 \geometry{
 a4paper,
 total={175mm,260mm},
 left=15mm,
 top=15mm,
 }

\usepackage{fontspec}
\usepackage{graphicx}
\usepackage{hyperref,babel}
\usepackage[cm]{fullpage}
\usepackage{fancyvrb}

\pagestyle{empty}

\begin{document}
\begin{quiz}{Wireshark-Introduzione-1}

\begin{multi}[points=1,multiple]{Intro-1 Lab: Q01 Introduzione e protocolli presenti nel PCAP.}
\textbf{Intro-1 Lab: Q01 Introduzione e protocolli presenti nel PCAP.}

Le risposte alle domande in questo modulo si basano sui pacchetti nel file \emph{intro-wireshark-trace1.pcap}.

Quali dei seguenti protocolli appaiono (cioè sono elencati nella colonna ``protocollo'' di Wireshark) nel PCAP?

\item[fraction=33.33333] TCP
\item QUIC
\item[fraction=33.33333] HTTP
\item DNS
\item UDP
\item[fraction=33.33333] TLSv1.2
\end{multi}

\begin{shortanswer}[points=1]{Intro-1 Lab: Q02 Tempo di risposta HTTP.}
\textbf{Intro-1 Lab: Q02 Tempo di risposta HTTP.}
Quanto tempo è trascorso dal momento in cui il messaggio HTTP GET è stato inviato fino a quando è stato ricevuto il messaggio HTTP OK di risposta?

Per impostazione predefinita, il valore della colonna ``Tempo'' nella finestra di elencazione dei pacchetti è l'ammontare di tempo, in secondi, trascorso dal momento in cui è iniziato la cattura di Wireshark. Inserisci la tua risposta nella forma "0,xyz" (senza virgolette) dove ``xyz'' rappresenta le prime tre cifre decimali del tempo di risposta HTTP. Calcola la tua risposta usando sei cifre decimali di precisione temporale, e poi inserisci la tua risposta arrotondata (non troncata) a tre cifre decimali:

\item 0,029
\end{shortanswer}

\begin{shortanswer}[points=1]{Intro-1 Lab: Q03.1 Indirizzo IP del server.}
\textbf{Intro-1 Lab: Q03.1 Indirizzo IP del server.}
Qual è l'indirizzo Internet (IP) del server web gaia.cs.umass.edu (noto anche come www-net.cs.umass.edu)?

Inserisci l'indirizzo in notazione decimale puntata del tipo ``w.x.y.z,'' dove ``w, x, y'' e ``z'' sono interi compresi tra 0 e 255. Ometti zeri iniziali (tranne se il valore di ``w, x, y'' o ``z'' è zero); include i punti nella tua risposta:

\item 128.119.245.12
\end{shortanswer}

\begin{shortanswer}[points=1]{Intro-1 Lab: Q03.2 Indirizzo IP del cliente.}
\textbf{Intro-1 Lab: Q03.2 Indirizzo IP del cliente.}
Qual è l'indirizzo Internet (IP) del cliente che ha inviato la richiesta HTTP al server gaia.cs.umass.edu?

Inserisci l'indirizzo del cliente in notazione decimale puntata del tipo ``w.x.y.z,'' dove ``w, x, y'' e ``z'' sono interi compresi tra 0 e 255. Ometti zeri iniziali (tranne se il valore di ``w, x, y'' o ``z'' è zero); include i punti nella tua risposta:

\item 10.0.0.44
\end{shortanswer}

\begin{multi}[points=1]{Intro-1 Lab: Q04 Quale web browser?}
\textbf{Intro-1 Lab: Q04 Quale web browser?}
Espandi le informazioni sul messaggio HTTP nella finestra ``Dettagli del pacchetto selezionato'' di Wireshark (vedi Figura 3 nella descrizione del laboratorio). Quale tipo di web browser ha emesso la richiesta HTTP? La risposta è mostrata alla fine delle informazioni relative al campo "User-Agent:".

\item Safari
\item* Firefox
\item Microsoft Edge
\item Nessuna di queste risposte.
\end{multi}

\begin{multi}[points=1]{Intro-1 Lab: Q05 Numero di porta.}
\textbf{Intro-1 Lab: Q05 Numero di porta.}
Espandi le informazioni sul Protocollo TCP nella finestra ``Dettagli del pacchetto selezionato'' di Wireshark (vedi Figura 3 nella descrizione del laboratorio) in modo da poter vedere i campi nel segmento TCP che trasporta il messaggio HTTP.

Qual è il numero di porta di destinazione (il numero che segue "Dest Port:" per il segmento TCP che contiene la richiesta HTTP) a cui questa richiesta HTTP viene inviata?

\item* 80
\item 23
\item 242
\item 53962
\end{multi}

\end{quiz}
\end{document}
